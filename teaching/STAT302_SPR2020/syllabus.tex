\documentclass[12pt]{article}

\usepackage[margin=1.25in]{geometry}
\usepackage{graphicx}
\graphicspath{{../charts/}}
\usepackage{amsmath}
\usepackage{amsfonts}
\usepackage{amsthm}
\usepackage{algorithm}
\usepackage{algpseudocode}
\usepackage[shortlabels]{enumitem}
\usepackage{color}
\usepackage{listings}
\lstset{breaklines=true}
\usepackage{appendix}
\usepackage[colorlinks]{hyperref}
\usepackage{subcaption}
\usepackage{placeins}
\usepackage[normalem]{ulem}

\newtheorem{theorem}{Theorem}
\newtheorem{proposition}{Proposition}

\newcommand{\Prob}{\text{P}}
\newcommand{\E}{\text{E}}
\newcommand{\Var}{\text{Var}}
\newcommand{\Cov}{\text{Cov}}
\newcommand{\sd}{\text{sd}}
\newcommand{\KL}{\text{KL}}
\newcommand{\I}{\text{I}}

\newcommand{\Norm}{\text{N}}
\newcommand{\Unif}{\text{U}}
\newcommand{\Pois}{\text{Pois}}
\newcommand{\Cauchy}{\text{Cauchy}}
\newcommand{\Laplace}{\text{LaPlace}}

\newcommand{\Ind}{\textbf{1}}
\newcommand{\sign}{\text{sign}}
\newcommand{\Mod}[1]{\ (\mathrm{mod}\ #1)}
\newcommand{\indep}{\rotatebox[origin=c]{90}{$\models$}}
\newcommand{\iid}{\overset{\text{i.i.d.}}{\sim}}

\newcommand{\Rr}{\mathbb{R}}
\newcommand{\Ff}{\mathbb{F}}
\newcommand{\Pp}{\mathbb{P}}
\newcommand{\Qq}{\mathbb{Q}}

\newcommand{\Ascript}{\mathcal{A}}
\newcommand{\Gscript}{\mathcal{G}}
\newcommand{\Rscript}{\mathcal{R}}
\newcommand{\Hscript}{\mathcal{H}}
\newcommand{\Xscript}{\mathcal{X}}
\newcommand{\Yscript}{\mathcal{Y}}
\newcommand{\Oscript}{\mathcal{O}}
\newcommand{\Pscript}{\mathcal{P}}
\newcommand{\Mscript}{\mathcal{M}}

\title{Syllabus: STAT 302 SPR 2020}
\author{Sheridan Grant\\(\textit{not yet} a Doctor; call me ``Sheridan'' or ``Prof. Grant'')\\\href{mailto:slgstats@uw.edu}{slgstats@uw.edu}}
\date{March 26, 2020}

\begin{document}\sloppy

\maketitle

\section*{Logistics}

\begin{itemize}
	\item \textbf{First class---on Zoom:} Monday, March 30, 3:30-4:50pm Pacific time.
	\item Zoom lectures: Monday/Wednesday 3:30-4:50pm Pacific time.
	\item Zoom recordings: posted a couple hours after class.
	\item Zoom office hours (\href{https://washington.zoom.us/j/7876861762?pwd=Nld6U2cvMWlUOVJUZ1B1UWVUbkFBQT09}{link to Zoom room}): Mondays and Fridays 1-3pm Pacific time (except Monday, March 30); Thursdays 10pm-12am Pacific time. There's also the discussion board (below)!
	\item Communications: any questions about homework, course content, or math/stats/programming in general should be asked on the class \href{https://piazza.com/washington/spring2020/stat302}{Piazza}. \textbf{I will not answer these questions by email.} I will check Piazza every weekday and probably most weekend days. Personal questions (``you miscalculated my points on this homework''; ``Can I an extension on the final project due to an illness''; etc.) should be emailed.
	\item \href{https://sheridanlgrant.github.io/teaching/STAT302_SPR2020}{Course website} will host links for Wednesday active lectures, lecture recordings, homework assignments, lecture slides, syllabus, helpful links---course info in general.
	\item \href{https://piazza.com/washington/spring2020/stat302}{Piazza} discussion board for homework and project questions (checked at least once every weekday).
	\item \href{https://canvas.uw.edu/}{Canvas} for checking grades and uploading homeworks. \textbf{Zoom meeting password is on the Canvas homepage. Please do not give this link to anyone not in the class.}
	\item \href{https://info201.github.io/}{UW iSchool Textbook} if needed for reference
	\item The Stat Tutoring Center has 1-on-1 appointments available. \href{https://www.stat.washington.edu/academics/tutoring}{Sign up here.}
\end{itemize}

\section*{Covid-19 (Coronavirus) and Online Classes}

Might as well address the elephant in the room: coronavirus and the ensuing online spring quarter. My goals for the course with respect to coronavirus are twofold: first, I will respect the difficulties that coronavirus and online learning impose on students. Some of you may be living in far-away time zones; for this reason, all lectures will be recorded and lecture attendance is optional (though you will need to earn participation credit for the Wednesday lectures as detailed below). If you become ill or for some other reason related to coronavirus, economic hardship, etc. cannot keep up with lectures or submit an assignment on time, I will work with you to ensure that you still earn the credit you deserve (and learn the material). The grading scheme for this class is designed to make earning credit straightforward even if you do not do as well on some parts of the course as you ordinarily would. You can think of grades as gravitating towards a 3.2: I do not want anyone that is working hard to not receive credit for the class due to the strange circumstances surrounding coronavirus, but you will also need to demonstrate true mastery of the material to earn grades close to a 4.0.

Second, this is still a University of Washington statistics class, and I do not intend to make it easy even though it is online and the coronavirus epidemic is a legitimate distraction. My hope is that everyone in the course receives credit, and the course design is flexible enough for anyone who makes an earnest effort to receive credit. But the class will take time and effort---programming is hard, statistics is hard, and you will need to engage with material frequently to learn it. I recommend following along with any code run in the lectures on your own machine, and taking notes on things you find difficult or important so that you can go back and try to understand them better later.

\section*{Prerequisites}

\begin{itemize}
	\item Courses: either STAT 311/ECON 311 or STAT 390/MATH 390
	\item \textbf{No prior programming experience is assumed.} If you've programmed before, the first couple weeks of the course might be a little slow for you. \textbf{Anyone can learn to program, regardless of gender, race, age, or disabilities. Programming is hard. It may not come easily for you---but you can do it.}
	\item While this course is primarily about programming, it is also about data science and statistics. You will be expected to know the material in the prerequisite course, and will be required to write analyses, interpret hypothesis tests and estimates, and use good statistical sense.
\end{itemize}

\section*{Learning Goals}

Affective Learning Goals: how is our relationship with programming/math/stats?
\begin{itemize}
	\item We will come to see success in programming (and math and statistics) as a product of \textbf{effort rather than ``natural ability.''} There's no such thing as ``math people'' and ``non-math people.''
	\item We will discover that solving programming problems is a creative, rather than procedural, process.
	\item We will develop resilience in problem solving by allowing failure of an initial approach to motivate other approaches that will succeed.
\end{itemize}

Cognitive Learning Goals: what will we learn about programming/math/stats as a whole?
\begin{itemize}
	\item We will write \textbf{elegant and generalizable code}. Code is written to be read and used by others, and extended/revised by yourself and others.
	\item We will write \textbf{robust code}, by thoroughly testing and trying to ``break'' our own code. Similarly, we will do \textbf{robust data science}, by trying to understand when our models behave as we expect and when they do not.
\end{itemize}

\section*{Homework}

There will be \textbf{two} weekly assignments. The first (``Completion'') will be assigned on Monday \textbf{(except for the first Monday, March 30)}, along with the recorded lecture, and due before Wednesday's class (3:30pm Pacific time). These will be \textbf{a completion grade:} 0 = did not submit, 1 = low effort or mostly incorrect, 2 = clear effort and at least somewhat correct. They are designed to make sure you watched and followed the Monday recorded lecture, and should be easy credit.

The second (``Graded'') will be assigned Wednesday after lecture and due the following Monday by 11:59pm Pacific time. This assignment will be longer and graded.

Both assignments will be uploaded on Canvas under the associated assignment.

\section*{Grading}

In each Wednesday lecture, I will ask ``active lecture chat'' questions, and expect responses in the Zoom chat \textbf{from those in attendance only.} I will also ask related ``active lecture email'' questions, which \textbf{should be answered only by people who did not attend active lecture in person} and who are watching the recording instead. The ``active lecture email'' responses are due Fridays by 3:30pm. You will receive 2 points for this ``Active Lecture Participation'' if your responses are complete and demonstrate effort; 1 point if your responses are incomplete or demonstrate little effort; and 0 points if you do not respond at all.

I will calculate your final average as 
\begin{align*}
&0.15 (\text{Active Lecture Participation}) + 0.2 (\text{Completion Homework Average}) \\ + &0.4 (\text{Graded Homework Average}) + 0.25 (\text{Final Project Grade}).
\end{align*}
I will not detail exactly how this final average maps to the GPA scale, but I expect $\approx 70\%$ to be the cutoff for a 3.0. \textbf{The purpose of giving you this formula is for you to understand the relative importance of the coursework:} $1/3$ of the course credit is merely completion, so if you participate actively in lectures and keep up with the completion homeworks, you have a great ``cushion'' for the $2/3$ of the final average that is graded.

\section*{Academic Integrity (Cheating)}

\textbf{I will not hesitate to report academic dishonesty to UW and/or penalize you via your grade.} Academic integrity in an online programming-based course is tricky. Here are the rules:
\begin{itemize}
	\item You may work with others in the class, but you may not copy any part a classmate's code. For example, if you figure out how to solve part of a problem, you may call your classmate and discuss how to solve it, or link them to a website you found helpful, but \textbf{you may not send them your code}. If you are working together on Zoom or in person and can see your partner's code, that's okay, but you still must write your own code. For very simple programming problems, it may be the case that most people have the same code, but for more complex problems everyone's code will be slightly different. \textbf{If you work on a Graded assignment with a partner:}, you must put the comment \verb|# Partner: partner's name| near the beginning of your submitted code.
	\item If you find code on the internet that helps you with a problem (this will probably happen frequently), you may copy it, but \textbf{you must alter it to suit the problem and code you have already written}. So if the code you found names the data table \verb|dat| and your code names it \verb|stocks|, rename the data table in the copied code \verb|stocks|.
	\item All non-code writing must be your own.
\end{itemize}

\section*{Software Downloads}

In this class, we will use the R statistical programming language, and the RStudio IDE (Integrated Development Environment--it's an application for writing and running code). You need to get these two things installed ASAP. Here's how to do it:

\begin{enumerate}
	\item \href{https://cran.rstudio.com/}{Download R} (if you haven't already):
	\begin{itemize}
		\item If you have a Windows machine, click ``Download R for Windows," then click ``install R for the first time," then click ``Download R 3.6.3 for Windows," then open and run the .exe file that is downloaded. Accept the default options for the installation.
		\item If you have a Mac, click ``Download R for (Mac) OS X," then click ``R-3.6.3.pkg," open that file and install with the default options.
		\item If you run into a roadblock, try to work your way around it, and if you cannot, make a note of what the problem is so I can help you. One common problem is not having administrator access to your machine, perhaps because you are using a laptop you do not own. If this is a case, you need to get the administrator username and password, or you need to get a different machine. It is crucial to be able to install software on your own machine!
	\end{itemize}
	\item \href{https://rstudio.com/products/rstudio/download/#download}{Download RStudio} (if you haven't already): there should be a big button that says ``Download RStudio for [your operating system]." If it has your operating system (OS) correct, click it. If not, scroll down and download the appropriate installer for your OS. Run the installer, and accept the default options.
\end{enumerate}

\section*{Miscellaneous}

\begin{itemize}
	\item Recommendation letter policy: you may want a recommendation letter from me. First, note that a recommendation letter from a PhD student carries much less weight than a letter from a professor, so for your own sake consider asking a professor first. Second, you will need to both do very well in the class (3.8+ GPA) and demonstrate interest in the subject for me to be able to write you an effective letter.
	\item DRS: if you have established accommodations with Disability Resources for Students (DRS), please let me know ASAP. If you feel you need special accommodations but have not yet established accommodations with DRS, contact me or DRS directly. DRS can help accommodate conditions including, but not limited to: mental health, attention-related, learning, vision, hearing, physical, or other conditions. DRS can be contacted at 206-543-8924, \href{mailto:uwdrs@uw.edu}{uwdrs@uw.edu}, or their \href{http://depts.washington.edu/uwdrs/}{website}.
\end{itemize}

\end{document}