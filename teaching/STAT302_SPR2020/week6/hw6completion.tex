\documentclass[12pt]{article}

\usepackage[margin=1.25in]{geometry}
\usepackage{graphicx}
\graphicspath{{../charts/}}
\usepackage{amsmath}
\usepackage{amsfonts}
\usepackage{amsthm}
\usepackage{algorithm}
\usepackage{algpseudocode}
\usepackage[shortlabels]{enumitem}
\usepackage{color}
\usepackage{listings}
\lstset{breaklines=true}
\usepackage{appendix}
\usepackage[colorlinks]{hyperref}
\usepackage{subcaption}
\usepackage{placeins}

\newtheorem{theorem}{Theorem}
\newtheorem{proposition}{Proposition}

\newcommand{\Prob}{\text{P}}
\newcommand{\E}{\text{E}}
\newcommand{\Var}{\text{Var}}
\newcommand{\Cov}{\text{Cov}}
\newcommand{\sd}{\text{sd}}
\newcommand{\KL}{\text{KL}}
\newcommand{\I}{\text{I}}

\newcommand{\Norm}{\text{N}}
\newcommand{\Unif}{\text{U}}
\newcommand{\Pois}{\text{Pois}}
\newcommand{\Cauchy}{\text{Cauchy}}
\newcommand{\Laplace}{\text{LaPlace}}
\newcommand{\Binom}{\text{Binom}}

\newcommand{\Ind}{\textbf{1}}
\newcommand{\sign}{\text{sign}}
\newcommand{\Mod}[1]{\ (\mathrm{mod}\ #1)}
\newcommand{\indep}{\rotatebox[origin=c]{90}{$\models$}}
\newcommand{\iid}{\overset{\text{i.i.d.}}{\sim}}

\newcommand{\Rr}{\mathbb{R}}
\newcommand{\Ff}{\mathbb{F}}
\newcommand{\Pp}{\mathbb{P}}
\newcommand{\Qq}{\mathbb{Q}}

\newcommand{\Ascript}{\mathcal{A}}
\newcommand{\Gscript}{\mathcal{G}}
\newcommand{\Rscript}{\mathcal{R}}
\newcommand{\Hscript}{\mathcal{H}}
\newcommand{\Xscript}{\mathcal{X}}
\newcommand{\Yscript}{\mathcal{Y}}
\newcommand{\Oscript}{\mathcal{O}}
\newcommand{\Pscript}{\mathcal{P}}
\newcommand{\Mscript}{\mathcal{M}}

\title{Homework 6 Completion}
\author{Sheridan Grant}
\date{Must be uploaded to Canvas under ``Homework 6 Completion'' by \textbf{Wednesday, May 6 at 3:30pm}}

\begin{document}\sloppy

\maketitle

\section*{Instructions}

Format your .RMD file using the template on the \href{https://sheridanlgrant.github.io/teaching/STAT302_SPR2020}{course website}. \textbf{Submit the .RMD file, the knitted .html output, and any other files or folders needed as a single .zip file.}

The grader will be compiling your .RMD file and making sure it knits. Any libraries/packages needed should be near the top of the .RMD file, so the grader can make sure they're installed. Any other files needed to knit the .html should be in the zipped folder you turn in.

Any time I ask you to demonstrate something, show something, generate something, etc., you must provide the code and/or text commentary that does so.

\section{Programming Puzzles}

This assignment will use the built-in \verb|mtcars| data set, in which the rows (observations) are car models and the columns are attributes. You will fit 4 linear models, each with number of cylinders as the predictor. For each outcome---either MPG (miles per gallon) or HP (horsepower)---fit 2 models, one with number of cylinders as a \verb|factor| type and one with number of cylinders as a \verb|numeric| type. That is, first treat number of cylinders as a real number that is linearly related to the outcome, and then treat it purely as an identifier of a group. Which type of model (factor or numeric) gives a better fit for each outcome, and why? For the MPG outcome, write a 1-sentence interpretation of the model coefficient(s) of interest for each type of model.

\end{document}