\documentclass[12pt]{article}

\usepackage[margin=1.25in]{geometry}
\usepackage{graphicx}
\graphicspath{{../charts/}}
\usepackage{amsmath}
\usepackage{amsfonts}
\usepackage{amsthm}
\usepackage{algorithm}
\usepackage{algpseudocode}
\usepackage[shortlabels]{enumitem}
\usepackage{color}
\usepackage{listings}
\lstset{breaklines=true}
\usepackage{appendix}
\usepackage[colorlinks]{hyperref}
\usepackage{subcaption}
\usepackage{placeins}

\newtheorem{theorem}{Theorem}
\newtheorem{proposition}{Proposition}

\newcommand{\Prob}{\text{P}}
\newcommand{\E}{\text{E}}
\newcommand{\Var}{\text{Var}}
\newcommand{\Cov}{\text{Cov}}
\newcommand{\sd}{\text{sd}}
\newcommand{\KL}{\text{KL}}
\newcommand{\I}{\text{I}}

\newcommand{\Norm}{\text{N}}
\newcommand{\Unif}{\text{U}}
\newcommand{\Pois}{\text{Pois}}
\newcommand{\Cauchy}{\text{Cauchy}}
\newcommand{\Laplace}{\text{LaPlace}}

\newcommand{\Ind}{\textbf{1}}
\newcommand{\sign}{\text{sign}}
\newcommand{\Mod}[1]{\ (\mathrm{mod}\ #1)}
\newcommand{\indep}{\rotatebox[origin=c]{90}{$\models$}}
\newcommand{\iid}{\overset{\text{i.i.d.}}{\sim}}

\newcommand{\Rr}{\mathbb{R}}
\newcommand{\Ff}{\mathbb{F}}
\newcommand{\Pp}{\mathbb{P}}
\newcommand{\Qq}{\mathbb{Q}}

\newcommand{\Ascript}{\mathcal{A}}
\newcommand{\Gscript}{\mathcal{G}}
\newcommand{\Rscript}{\mathcal{R}}
\newcommand{\Hscript}{\mathcal{H}}
\newcommand{\Xscript}{\mathcal{X}}
\newcommand{\Yscript}{\mathcal{Y}}
\newcommand{\Oscript}{\mathcal{O}}
\newcommand{\Pscript}{\mathcal{P}}
\newcommand{\Mscript}{\mathcal{M}}

\title{Homework 1 Completion\\\textbf{SUGGESTED, NOT DUE, DO NOT TURN IN}}
\author{Sheridan Grant}
\date{March 30, 2020}

\begin{document}\sloppy

\maketitle

\section{Kooky Kalculations}

Build up to a tricky calculation by combining different bits of R code.
\begin{enumerate}
	\item The ``modulus'' operator is \verb|%%|. The code \verb|x %% y| outputs the remainder after dividing \verb|x| by \verb|y|. For example, \verb|18 %% 5| outputs \verb|3|. For what positive integers \verb|x| is \verb|x %% y| equal to 0 \textit{only} if \verb|y == x| or \verb|y == 1|? Find some examples. Do these numbers have a name?
	\item You can do operations like \verb|+,-,*,/| with lists. See what happens when you run \verb|(1:100) + 5|. Then run \verb|(1:100) %% 2|. What does this code have the computer figure out about the numbers from 1 to 100?
	\item Write code to output a vector of ``logicals'' (\verb|TRUE|s and \verb|FALSE|s) where the $n$th logical is \verb|TRUE| if $n$ is \textit{even} (hint: your solution will probably use \verb|!|).
	\item Read about the \verb|which| function. Write a line of code that finds the indices of the even integers from 1 to 100 (i.e. outputs \verb|2 4 6 8| etc.).
	\item You can ``multi-index'' vectors. For example, \verb|(3:8)[5]| outputs \verb|7|, but what does \verb|(3:8)[c(1,3,5)]| output?
	\item Using everything you learned above, write a line of code that computes the sum of the even numbers from 1 to 100.
\end{enumerate}

\end{document}